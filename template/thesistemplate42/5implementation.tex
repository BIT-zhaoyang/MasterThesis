\chapter{Experiment}
\label{chapter:Experiment}

% what data are used? (period time)
% what preprocess was done? (remove data generated by system)
% how to represent the data? (time sequence)
% how to measure the distance? (modified edit distance)
% Which method to use, why? ( DBSCAN, MM)
	% Can't compute a mean value,	there should be clusters in the data, clustered by department
	% MM is enough. No need to use HMM to complicate the model. Using HMM doesn't explain intuitively, error happened in one phase doesn't lead to error in
	%	the following phase. Different visits don't have direct impact either. Error are usually caused by operators, at least I assume it is this case.
	
This chapter describes the detailed experimental implementation. Following topics will be discussed:
\begin{itemize}
	\item	Which specific data are used? How to represent the data? What preprocess has been done? What related functions are defined?
	\item	Which methods are used? What's the reason to use such methods? 
\end{itemize}
The three topics will be discussed in three sections, respectively. Analysis of the result will be postponed to Chapter~\ref{chapter:evaluation}.

\section{Data Details}
Since the akseli system has gone through several updation after its first release, many features have changed, including how the visit log is reocrded. Considering this aspect, only data generated after year 2014 is retrieved. All the data are from Oulu Hospital, with patient privacy information eliminated. In total, 243K unique visits with 1.93 million entries are retrieved, which spans from January to July. The retrieved data consists of three columns: visit id, event type, and recorded time of the event. Other information, such as which resource generated the data, is not retrieved. Among the retrieved data, 7 event types are adopted. The used event types are: \texttt{ENROLLING}, \texttt{WAITING}, \texttt{IN\_TREATMENT\_ROOM}, \texttt{PAUSED}, \texttt{IN\_TREATMENT\_ROOM\_FROM\_PAUSED}, \texttt{CLOSED}, and \texttt{CANCELLED}.
Inspired by Huang et al~\cite{huang2012anomaly}, a similiar representation form is proposed. In this representation, each patient visit consists of several pairs. Each pair is a combination of event type and duration of being in this event.% Write what is the difference between my representation and huang's. And why is it?